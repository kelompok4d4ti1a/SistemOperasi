Nama kelompok
1.Duvan Silalahi
2.Ilham Habibi
3.Damara Benedikta
4.Muhammad Fahmi
5.Oni Waldus
\section{frekuensi serial}
Frekuensi adalah ukuran jumlah putaran ulang per peristiwa dalam satuan detik dengan satuan Hz.
Pada frekuensi serial, setiap saat hanya dibutuhkan 1 bit data yang akan dikirim. Dengan kata lain, bit data akan dikirim satu per satu. frekuensi ini memiliki keuntungan yang hanya membutuhkan satu jalur dan beberapa kabel dibandingkan dengan komunikasi paralel. Pada prinsipnya frekuensi serial adalah frekuensi di mana transmisi data dilakukan per bit sehingga lebih lambat daripada frekuensi paralel, atau dengan kata lain frekuensi serial adalah salah satu metode komunikasi data dimana hanya satu bit data yang dikirimkan melalui seutas kabel pada waktu tertentu.
Untuk menghitung frekuensi, seseorang menetapkan jarak waktu, menghitung jumlah kejadian peristiwa, dan membagi hitungan ini dengan panjang jarak waktu. Pada Sistem Satuan Internasional, hasil perhitungan ini dinyatakan dalam  satuan hertz (Hz).
Untuk menghitung frekuensi, seseorang menetapkan jarak waktu, menghitung jumlah kejadian peristiwa, dan membagi hitungan ini dengan panjang jarak waktu. Pada Sistem Satuan Internasional, hasil perhitungan ini dinyatakan dalam  satuan hertz (Hz).