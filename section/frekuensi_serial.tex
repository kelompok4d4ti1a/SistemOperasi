Kelompok 4

\section {Frekuensi Serial}
Sebelum kita mengetahui apakah itu frekuensi serial, kita harus tau apakah defenisi dari frekuensi.
\subsection  {defenisi frekuensi}
Frekuensi adalah ukuran jumlah putaran ulang per peristiwa dalam satuan detik dengan satuan Hz.
Untuk menghitung frekuensi, seseorang menetapkan jarak waktu, menghitung jumlah kejadian peristiwa, dan membagi hitungan ini dengan panjang jarak waktu. Pada Sistem Satuan Internasional, hasil perhitungan ini dinyatakan dalam satuan hertz (Hz) yaitu nama pakar fisika Jerman Heinrich Rudolf Hertz yang menemukan fenomena ini pertama kali. Frekuensi sebesar 1 Hz menyatakan peristiwa yang terjadi satu kali per detik.
