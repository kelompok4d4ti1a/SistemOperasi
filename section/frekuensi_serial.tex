Kelompok 4

\section {Frekuensi Serial}
Sebelum kita mengetahui apakah itu frekuensi serial, kita harus tau apakah defenisi dari frekuensi.
\subsection  {defenisi frekuensi}
Frekuensi adalah ukuran jumlah putaran ulang per peristiwa dalam satuan detik dengan satuan Hz.
Untuk menghitung frekuensi, seseorang menetapkan jarak waktu, menghitung jumlah kejadian peristiwa, dan membagi hitungan ini dengan panjang jarak waktu. Pada Sistem Satuan Internasional, hasil perhitungan ini dinyatakan dalam satuan hertz (Hz) yaitu nama pakar fisika Jerman Heinrich Rudolf Hertz yang menemukan fenomena ini pertama kali. Frekuensi sebesar 1 Hz menyatakan peristiwa yang terjadi satu kali per detik.
\section {Frekuensi Serial}
Pada frekuensi serial, setiap saat hanya dibutuhkan 1 bit data yang akan dikirim. Dengan kata lain, bit data akan dikirim satu per satu. frekuensi ini memiliki keuntungan yang hanya membutuhkan satu jalur dan beberapa kabel dibandingkan dengan komunikasi paralel. Pada prinsipnya frekuensi serial adalah frekuensi di mana transmisi data dilakukan per bit sehingga lebih lambat daripada frekuensi paralel, atau dengan kata lain frekuensi serial adalah salah satu metode komunikasi data dimana hanya satu bit data yang dikirimkan melalui seutas kabel pada waktu tertentu.
\section {Frekuensi serial adalah frekuensi yang pengiriman datanya per-bit secara berurutan dan bergantian. Frekuensi ini mempunyai suatu kelebihan yaitu hanya membutuhkan satu jalur dan kabel yang sedikit dibandingkan dengan komunikasi paralel. Pada prinsipnya komunikasi serial merupakanfrekuensi dimana pengiriman data dilakukan per bit sehingga lebih lambat dibandingkan frekuensi parallel, atau dengan kata lain komunikasi serial merupakan salah satu metode komunikasi data di mana hanya satu bit data yang dikirimkan melalui seuntai kabel pada suatu waktu tertentu. Pada dasarnya komunikasi serial adalah kasus khusus komunikasi paralel dengan nilai n = 1, atau dengan kata lain adalah suatu bentuk komunikasi paralel dengan jumlah kabel hanya satu dan hanya mengirimkan satu bit data secara simultan.
\subsection {Hal ini dapat disandingkan dengan komunikasi paralel yang sesungguhnya di mana n-bit data dikirimkan bersamaan, dengan nilai umumnya 8 ≤ n ≤ 128.