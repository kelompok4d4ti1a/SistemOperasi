Kelompok 4

\section {Frekuensi Serial}
Sebelum kita mengetahui apakah itu frekuensi serial, kita harus tau apakah defenisi dari frekuensi.
\subsection  {defenisi frekuensi}
Frekuensi adalah ukuran jumlah putaran ulang per peristiwa dalam satuan detik dengan satuan Hz.
Untuk menghitung frekuensi, seseorang menetapkan jarak waktu, menghitung jumlah kejadian peristiwa, dan membagi hitungan ini dengan panjang jarak waktu. Pada Sistem Satuan Internasional, hasil perhitungan ini dinyatakan dalam satuan hertz (Hz) yaitu nama pakar fisika Jerman Heinrich Rudolf Hertz yang menemukan fenomena ini pertama kali. Frekuensi sebesar 1 Hz menyatakan peristiwa yang terjadi satu kali per detik.
\section {Frekuensi Serial}
Pada frekuensi serial, setiap saat hanya dibutuhkan 1 bit data yang akan dikirim. Dengan kata lain, bit data akan dikirim satu per satu. frekuensi ini memiliki keuntungan yang hanya membutuhkan satu jalur dan beberapa kabel dibandingkan dengan komunikasi paralel. Pada prinsipnya frekuensi serial adalah frekuensi di mana transmisi data dilakukan per bit sehingga lebih lambat daripada frekuensi paralel, atau dengan kata lain frekuensi serial adalah salah satu metode komunikasi data dimana hanya satu bit data yang dikirimkan melalui seutas kabel pada waktu tertentu.
\section {serial mode}
Mode serial memerlukan sinkronisasi atau penyesuaian yang berfungsi untuk:
\subsection {fungsi sinkronisasi}
Mengetahui kapan sinyal yang diterimanya adalah sedikit data (bit sinkronisasi)
Mengetahui kapan sinyal yang diterimanya membentuk karakter (sinkronisasi karakter)
Mengetahui kapan sinyal yang diterimanya membentuk blok data (memblokir sinkronisasi)
Selanjutnya, transmisi serial dapat mengambil bentuk dua jenis, yaitu transmisi serial sinkron (sinkron) dan transmisi serial asynchronous (asynchronous). Berikut ini adalah penjelasan dari masing-masing jenis transmisi serial. Transmisi Serial Sinkron.